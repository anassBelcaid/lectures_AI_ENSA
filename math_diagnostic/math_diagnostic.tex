\documentclass{beamer}
\usetheme{ensam}
\usepackage{pgfplots}
\usepackage{subcaption}
\usepackage{acronym}
\usepackage{tikz}
\usetikzlibrary{calc}
\usepackage{amsmath}
\usepackage {algorithmic}
\usepackage{algorithm}
\usepackage{eqparbox}
\usepackage[font=scriptsize]{caption}
\usetikzlibrary{bayesnet,positioning,calc}
\tikzstyle{obs} = [latent,fill=lightBlue]
\tikzstyle{default}=[draw=sexyRed,thick,rounded corners,text width=0.5in,font=\scriptsize,align=center]
\usepgfplotslibrary{colorbrewer}
\definecolor{ForestGreen}{RGB}{34,139,34}
\newcommand{\comment}[1]{\textcolor{ForestGreen}{#1}}
%algorithmic comment
\renewcommand\algorithmiccomment[1]{%
  \hfill\comment{\#\scriptsize\eqparbox{COMMENT}{#1}}%
}
\renewcommand{\algorithmicrequire}{\textbf{Input:}}
\renewcommand{\algorithmicensure}{\textbf{Output:}}
\title{Math Self Diagnostique}
\author{\underline{A.Belcaid}}
\institute{\small ENSA-Fès} 

%tikz bayesian theme
\usetikzlibrary{bayesnet,positioning,calc}
\tikzstyle{obs} = [latent,fill=lightBlue]
\tikzstyle{default}=[draw=sexyRed,thick,rounded corners,text width=0.5in,font=\scriptsize,align=center]
\DeclareMathOperator{\argmin}{argmin}

\pgfplotsset{every tick label/.append style={font=\tiny}}



%acronyms


% add bibliography
\usepackage[style=authoryear]{biblatex}
\renewcommand*{\nameyeardelim}{\addcomma\addspace}

\begin{document}
\maketitle

\begin{frame}
\tableofcontents
\end{frame}


\section{Espérance}%
\label{sec:esperance}

\begin{frame}[<+->]{Espérance}
  
  \begin{block}{Enoncé}
  Une personne lance un \alert{ \textbf{Dé} } à six faces. Vous gagner le
  nombre points affiché par le dé.
\end{block}

\vspace*{1cm}

  \begin{enumerate}
    \item Quelle est \textbf{l'espérance} des points gagnés pour une seul lancement. 
    \item Même Question après \structure{$\mathbf{2}$} lancements.
    \item Après \structure{$\mathbf{100}$}
  \end{enumerate}
\end{frame}



\section{probabilité jointe}%
\label{sec:probabilite_jointe}

\begin{frame}[t]{Probabilité jointe}
  
  \begin{block}{Enoncé}
   Pour deux distributions $X$ et $Y$, choisissez les formules correctes:
  \end{block}
  \pause
\begin{block}{}
  \begin{itemize}
    \item[$\square$] $P(x,y) = P(x)P(y)$
    \item[$\square$] $P(x,y) = P(x|y)P(y)$
    \item[$\square$] $P(x,y) = P(x|y)P(y|x)$
    \item[$\square$] $P(x) =\sum_y P(x|y)$
    \item[$\square$]  $ P(x) = \sum_y P(x,y)$
    \item[$\square$] Aucune formule
  \end{itemize} 
\end{block}
\end{frame}

\section{Probabilité conditionnelle}%
\label{sec:probabilite_conditionnelle}

\begin{frame}[t]{Probabilité conditionnelle}
  
  \begin{block}{Enoncé}
    Nous lançons \textbf{deux} Dé uniformes à $\mathbf{6}$ faces. 
  \end{block}

  \begin{block}{}
    \begin{enumerate}
      \item Calculer  la probabilité d'obtenir un \alert{\textbf{double}}.
      \item Sachant que le résultat obtenu est \alert{inférieur} à
        $\mathbf{4}$. Calculer la probabilité qu'un \textbf{double}  à
        été lancé.
    \end{enumerate}
  \end{block}
\end{frame}


\section{Systèmes d'équations linéaires}%
\label{sec:systemes_d_equations_lineaires}

\begin{frame}[t]{Équations linéaires}
  
  \begin{block}{Enoncé}
    Sachant que $x = (\dfrac{1}{2}) y + \dfrac{1}{2}(x+1)$ et que 
    $y = (\dfrac{1}{3})y + (\dfrac{1}{3})(x+2)

    \begin{itemize}
      \item Quelle est la valeur de $x$
      \item Quelle est la valeur de $y$
    \end{itemize}
  \end{block}

\end{frame}


\section{Logarithmes}%
\label{sec:logarithmes}

\begin{frame}[t]{Logarithmes}
  
  \begin{block}{Enoncé}
    Selectionner les formules correctes:
  \end{block}
  \begin{block}{}
    \begin{itemize}
      \item[$\square$]  $2^{xy} = 2^x 2^y$
      \item[$\square$] $2^{x+y}  = 2^x2^y$
      \item[$\square$] $2^{x+y} = 2^x + 2^y$
      \item[$\square$] $log(3^x) = log(3)log(x)$
      \item[$\square$] $log(3^x) = xlog(3)$
      \item[$\square$] $log(3^x) = xlog(3)$
      \item[$\square$] $log(3^x) = 3x$
      \item[$\square$] Aucune formule
    \end{itemize}
  \end{block}
\end{frame}

\section{Structure de données}%
\label{sec:structure_de_donnees}

\begin{frame}[t]{Structure de données}
  
  \begin{block}{Enconé}
    \small
    \begin{enumerate}
    \small
        \item Quelle est l'opération critique qui est plus rapide dans
          les \textbf{tables de hashage} que dans les \textbf{listes chainées}
          \begin{itemize}
            \scriptsize
            \item[$\square$] Insérer un élement.
            \item[$\square$] Tester l'existence d'un élement.
          \end{itemize}
        \item En moyenne, quelle est la \alert{\textbf{compléxité}} de
          cette
        opération dans une table de hashage.
          \begin{itemize}
            \scriptsize
            \item[$\square$] $\mathcal{O}(1)$
            \item[$\square$] $\mathcal{O}(n)$
            \item[$\square$] $\mathcal{O}(log(n))$
            \item[$\square$] $\mathcal{O}(n^2)$
            \item[$\square$] Aucne réponse
          \end{itemize}
      \item Quelle est la complexité de cette opération pour les listes
        chainnées:
          \begin{itemize}
            \scriptsize
            \item[$\square$] $\mathcal{O}(1)$
            \item[$\square$] $\mathcal{O}(n)$
            \item[$\square$] $\mathcal{O}(log(n))$
            \item[$\square$] $\mathcal{O}(n^2)$
            \item[$\square$] Aucne réponse
          \end{itemize}
    \end{enumerate}
  \end{block}
\end{frame}

\end{document}
